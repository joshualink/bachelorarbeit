\chapter{Grundlagen}\label{ch:grundlagen}
In diesem Kapitel wird das bestehende Probabilistic Scene Model - System vorgestellt und erklärt. Dabei wird auf das Grundprinzip, die Struktur und die einzelnen Komponenten eingegangen.
\section{PSM - Probabilistic Scene Model}
Bei der Entwicklung des Systems war das Ziel, ein System zur Erkennung von Szenen zu entwickeln. Dabei sollten viele Informationen umfasst werden. Dies umfasste die beteiligten Objekte bzw. deren Erscheinung, welche von den zur Erkennung eingesetzten Werkzeugen abhängig ist. Die Auftrittshäufigkeit der Objekte sollte ebenfalls mit einfließen. Der Hauptinformationsträger sind die Relationen der Objekte untereinander, welche in Form von relativen Objektlagen berücksichtigt wurden. Es wurde berücksichtigt, dass Relationen nicht statisch, sondern dynamisch sind. Da eine Szene auch unabhängig von ihrem Ort der Demonstration erkannt werden soll, musste die Lagebeschreibung invariant gegenüber Rotation und Translation sein.\smallskip\\
Weiterhin sollte Robustheit gegenüber verschiedenen Störfaktoren bestehen. Fehlende Objekte sollen eine Erkennung der Szene erlauben, wenn auch mit entsprechend reduzierter Konfidenz. Da jedes Objekt fehlen kann darf es kein zentrales Referenzobjekt geben, von dem die Relationen zu den anderen Objekten der Szene ausgehen. Überzählige Objekte sollen sich nicht negativ auf das Erkennungsergebnis auswirken. Generell soll sich die Funktionsweise des entwickelten Systems im Rahmen dessen bewegen, was plausibel erscheint.\smallskip\\
Es wird im folgenden erst auf die Struktur des Systems, dann auf die Eingabeverarbeitung, das innere Datenmodell und letztendlich auf die eigentliche Erkennung eingegangen.
\begin{deprecated}
\cite{gehrung14}
\end{deprecated}
\subsection{Aufbau}
Das System ist in mehrere Komponenten aufgeteilt die verschiedenen Aufgaben dienen. Für diese Arbeit wichtig sind vorallem die Komponenten Learner und Inference. Der Learner ist für das Anlernen der Daten und das Einspeichern von Szenen zuständig. Die Inference übernimmt die Erkennung der Szene indem sie in Echtzeit Daten empfängt und die Wahrscheinlichkeiten der Szenen ausgibt. Es gibt außerdem eine Visualizer-Komponente, die dafür zuständig ist, dass verschiedene programminterne Prozesse über Rviz für den Nutzer sichtbar und verständlich gemacht werden.
\subsection{Learner - Anlernen der Daten}
Die Learnerkomponente berechet aus den Objekten, die sie bekommt, die Parameter für das Szenenmodell, dass für die Erkennung benötigt wird. Der Learner ist in mehrere Teilbereiche eingeteilt, Engine, OCM und Szenenmodell. Engine ist die Schnittstelle des Learners und kapselt die anderen beiden Architekturgliederungen voneinander ab.\smallskip\\
Abbildung \ref{img:learnerclass} beschreibt den Learner als Klassendiagramm. Die darin vorkommende \textit{SceneLearningEnginge} ist die Schnittstelle des Moduls zum Rest des PSM- Systems. Sie liest die in der Launch-Datei gegebenen Parameter aus und überprüft, ob alle Parameter sich mit dem passenden Datentyp auslesen lassen. Außerdem ist die Klasse für Visualisierung des Modells zuständig, welches gerade gelernt wurde. \smallskip\\
Die gegebenen Objekte könnene mehrere Szenen beschreiben, da man auch Objekte aus einer Datei auslesen kann und diese jeweils eine pattern variable haben, die beschreibt zu welcher Szene sie zugehörig sind. Für jede Szene die gerade gelernt wird, gibt es einen Lerner in der \textit{SceneModelLearner}-Klasse, welche dafür verantwortlich ist, die einzelnen Lerner zu verwalten. Die Daten der verschiedenen Szenen werden auf die dazugehörigen Lerner aufgeteilt und es gibt eine seperaten Lerner für den Hintergrund, dessen Daten für den Fall wichtig sind, dass keine der Szenen in den gemessenen Objekten vertreten ist. Es ist sozusagen die Szene, die die Gegenwahrscheinlichkeit symbolisiert.\smallskip\\
Alle Lerner erben vom \textit{SceneLearner}, der eine abstrakte Basisklasse darstellt und eine Schnittstelle für das Lernen, Speichern und Visualisieren bietet. Der Lerner für den Hintergrund ist eine Instanz des \textit{BackgroundSceneLearner}, welcher blos die Anzahl der Objekte abspeichert, um daraus einen validen Schwellenwert beziehungsweise eine sinnvolle Gegenwahrscheinlichkeit zu bestimmen. Der \textit{ForegroundSceneLearner} ist eine abstrakte Klasse, sodass man das Modell der Berechnung leicht austauschen kann, allerdings konzentriert sich das PSM-System auf den Einsatz des OCM als Repräsentation der Szenenobjekte. Der \textit{OCMForegroundSceneLearner} ist eine Unterklasse des \textit{ForegroundSceneLearner} und kapselt die Lerner für den Vordergrund, welche Instanzen der Klasse \textit{OCMSceneObjectLearner} sind.
\begin{deprecated}
\cite{gehrung14}
\end{deprecated}
\begin{figure}
	\centering
	\includegraphics[width=15cm]{bilder/LearnerClass.pdf}
	\caption{Klassendiagramm des Learner-Moduls}
	\label{img:learnerclass}
\end{figure}
\subsection{Model - Szenenmodell}
Das Szenenmodell wird als XML-Datei abgespeichert. Dadurch kann man manuell das erstellte Modell lesen, verstehen und verändern, falls dies zum testen nötig ist. Man kann auch leicht mehrere erstellte Szenenmodelle kombinieren, da man alle Parameter verändern und leicht eine zusätzliche Szene aus einem anderen Modell hinzufügen oder ersetzen kann. \smallskip\\ 
Abbildung \ref{img:modelexample} zeigt ein Beispielmodell für eine Frühstücksszene namens \textit{breakfast} und die immer vorhandene Hintergrundszene \textit{background}. Jeder Szene ist eine apriori-Wahrscheinlichkeit zugeordnet, welche beschreibt, wie wahrscheinlich es grundsätzlich ist, dass die Szene auftritt, allerdings ist diese im PSM-System generisch gleichverteilt und kommt nur der Vollständigkeit halber vor. Man kann diese Werte falls nötig manuell ändern, das erstellte Szenenmodell der Learner-Komponente wird aber stets gleichverteilte Werte beinhalten. Der Teil der Hintergrundszene im Modell beinhaltet Informationen über die Anzahl unterschiedlicher Objekte und die Größe des Bereichs auf dem Szenen erkannt werden.\smallskip\\
Die Szene namens \textit{breakfast} beinhaltet zwei Szenenobjekte, \textit{Cup} und \textit{CoffeeBox}. Als Beispiel wurde eine einfache Szene mit geringer Objektanzahl gewählt um die Erklärung möglixhst simpel zu halten. Außerdem ist nur das Szenenobjekt \textit{Coffebox} ausgeklappt in der Modellabbildung, damit die Abbildung übersichtlicher ist. Für jeden Term des OCM sind die entsprechenden Parameter im Objekt gespeichert. Jedes Objekt in der Szene hat eine apriori-Wahrscheinlichkeit, welche die Wichtigkeit des Objekts in der Szene verdeutlicht. Desweiteren enthält es Informationen über die Anzahl der Slots, welche zwar aus dem kompletten Modell hergeleitet werden kann, man spart aber Ladezeit dadurch. Die Terme die im OCM vorkommen sind \textit{Scene Shape} und \textit{Objekt Appearance}.\smallskip\\
Der \textit{Scene Shape}- Term wird im Modell mit \textit{shape} bezeichnet und beinhaltet einen Baum, welcher die Relation zwischen den Szenenobjekten beschreibt. Der Baum besteht aus einem Wurzelknoten \textit{root} und beliebig vielen Kindsknoten \textit{child}, welche wiederum Kindsknoten beinhalten können. Der Wurzelknoten enthält das Volumen \textit{volume} auf dem die Erkennung arbeiten soll und die Kindsknoten jeweils Daten die die relative Lage zum direkten Elternknoten bezeichnen. Diese Daten, die unter der Bezeichnung \textit{pose} gespeichert sind, sind eingeklappt, da es unübersichtlich wäre, und enthalten Mittelwerte, Gewichte und Kovarianzmatrizen der einzelnen Gauss-Kernel.\smallskip\\
Der \textit{Objekt Appearance}-Term ist in zwei Abschnitte aufgeteilt. Erst wird jedem Objektnamen ein Index zugeordnet. Anschließend gibt es eine Wahrscheinlichkeitstabelle, die für jeden Index eine Auftrittswahrscheinlichkeit definiert. Der Index 0 wird bei der Zuordnung bewusst nicht belegt, da er ein Platzhalter für unbekannte Objekte ist.
\begin{deprecated}
\cite{gehrung14}
\end{deprecated}
\begin{figure}
	\centering
	\includegraphics[width=15cm]{bilder/Modell.pdf}
	\caption{Beispiel eines Szenenmodells - Frühstücksszene}
	\label{img:modelexample}
\end{figure}
\subsection{Inference - Szenenerkennung}
Die Szenenerkeennung oder auch Inferenz hat zur Aufgabe, aufgrund von dem gegebenen Szenenmodell und momentan erkannten Objekten, im folgenden auch Evidenz genannt, für jede Szene im Modell einzuschätzen, wie wahrscheinlich es ist, dass diese in der Evidenz vorkommt. Die verschiedenen Szenenwahrscheinlichkeiten werden dann miteinander verrechnet und es wird die relative Wahrscheinlichkeit für jede Szene ausgegeben, dass diese vorhanden ist. Diese Erkennung passiert in Echtzeit und ist als datengetriebenes Modul zu betiteln, da sich die Ausgaben und das Verhalten des Programms, maßgeblich dadurch verändert, welche Daten die Evidenz enthält, das heißt, welche Objekte erkannt werden.\smallskip\\
In Abbildung \ref{img:inferenceclass} sieht man das Klassendiagramm der Szenenerkennung, welches vergleichbar zu dem des Learner-Moduls strukturiert ist. Die verschiedenen Teilbereiche sind wieder Engine, OCM und Szenenmodell. Engine ist die Schnittstelle und kapselt Szenenmodell und OCM voneinander ab. \textit{SceneInferenceEngine} ist für das Auslesen der Parameter aus der Launch-Datei, für die Visualisierung des Erkenners und die Annahme und weitergabe der erkannten Evidenz zuständig. Diese wird an \textit{SceneModelDescription} weiter gegeben, welche für jede Szene im Modell eine Instanz der Klasse \textit{SceneDescription} erstellt. Ob es sich um eine Vorder- oder Hintergrundszene handelt, kann man dadurch erkennen, die daraufhin erstellte Instanz einer Unterklasse der Abstrakten Klasse \textit{SceneContent} entweder vom Klassentyp \textit{BackgroundSceneModel} oder \textit{ForegroundSceneModel} ist. \smallskip\\
In \textit{ForeGroundSceneModel} sind Instanzen von \textit{SceneObjektDescription} gekapselt, allerdings wird im PSM-System nur eine einzige Instanz benutzt, da dieses nur den Ansatz des OCM verfolgt und diese Kapselung dafür gedacht ist verschiedene Ansätze möglich zu machen. Die Unterklasse \textit{OCMSceneObjektContent} ist der besagt Algorithmus des PSM-Systems und beinhaltet mehrere \textit{TermEvaluator} die für die \textit{Object Appearance}, \textit{Object Existance} und \textit{SceneShape} zuständig sind.\smallskip\\
Die jeweils berechneten Wahrscheinlichkeiten werden eingespeichert und von der \textit{SceneInferenceEngine} über mehrere Klassen hinweg abgefragt und über das Visualisierungsmodul ausgegeben.
\begin{deprecated}
\cite{gehrung14}
\end{deprecated}
\begin{figure}
	\centering
	\includegraphics[width=15cm]{bilder/InferenceClass.pdf}
	\caption{Klassendiagramm des Inference-Moduls}
	\label{img:inferenceclass}
\end{figure}


