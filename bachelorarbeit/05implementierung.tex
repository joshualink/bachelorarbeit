\chapter{Implementierung}\label{ch:implementierung}

Im Kapitel Implementierung wird alles beschrieben und erklärt, was am bestehenden PSM-Projekt verändert und hinzugefügt wurde. Außerdem werden ausgewählte hinzugefügte und veränderte Klassen sowie launch-Dateien dokumentiert, sodass das Kapitel das Verständnis und die Nutzung der Neuheiten im System vereinfacht. Nachdem der Umbauprozess beschrieben wird, bei dem eine Klasse komplett aus dem PSM-Projekt ausgetauscht wurde, widmet sich das Kapitel der Umsetzung des Algorithmuskonzepts und der Einbettung in das vorhandene System.

\section{Umbau PSM}
Da das Paket "pbd\_msgs" nicht konstenlos zur Verfügung gestellt wird, mussten alle Vorkommen der Klassen aus diesem Paket zu alternativen Ersatzklassen geändert werden. Teilweise konnte man dies durch simple Ersetzung erreichen, allerdings gab es nicht für jede Klasse eine Ersatzklasse mit dem selben Funktionsumfang. In den Fällen, in denen Funktionen fehlten oder geringfügig anders funktionierten, konnte man den Umbau durch kleine Anpassungen erreichen oder musste eigene Funktionen schreiben, welche die nötigen Operationen verichten konnten. Alle auf diese Weise programmierten Funktionen wurden hinreichend auf Gleichheit mit ihren Ursprungsfunktionen in ihrer Funktionsweise getestet, indem die Ergebnisse bei gleichen Eingangsparametern abgeglichen wurden. Außerdem habe ich eine Datenbankschnittstelle für das PSM-System hinzugefügt.
\subsection{Klassenaustausch}

\subsection{Ersatzfunktionen}

\subsection{Datenbankeinbindung}
Im vergleichbaren ISM Projekt, welches eine ähnliche Zielsetzung wie das ISM Projekt hat, allerdings einen nicht stochastischen Ansatz verfolgt, kann man die Szenen, die das System lernen soll aus einer Datenbank auslesen. Aus Gründen der Vergleichbarkeit wie auch dem Komfort ist es sinnig, diese Datenbankschnittstelle für das PSM System nachzurüsten. Um dies zu erreichen wurde ein Datenbankpfad in die launch Datei des Learners(learner.launch) hinzugefügt und der veraltete rosbagfiles Parameter damit ersetzt. Die Datenbank wird innerhalb der SceneLearningEngine Klasse ausgelesen und die erhaltenen Daten, welche jeweils Vorkommen und von verschiedenen Objekten in diversen Szenen repräsentieren, werden anschließend konvertiert, sodass sie für das System sinnvolle AsrObject-Instanzen werden.\smallskip\\



\section{Differenzbasierter Erkennungsalgorithmus}
Einerseits beschreibt dieses Kapitel, wie der im Konzept vorgestellte Algorithmus programmiertechnisch umgesetzt, andererseits wie dieser in das bestehende System eingebunden wurde. 
\subsection{Algorithmus}
\subsection{Einbindung PSM}
Da davon abgesehen wird die Parametisierung des Szenenmodells zu nutzen, braucht man den Learner und das Szenenmodell nur noch zu dem Zweck, dass überliefert wird, welche Szenen potentiell erkannt werden sollen. Die einzigen Informationen die der algorithmus nutzt sind der Name der Szene und Anzahl der enthaltenen Objekte.
In der Launch Datei der Erkennung(inference.launch) wurde ein neuer Parameter hinzugefügt um den Pfad der Datenbank anzugeben. ...

alle Klassen die umgebaut wurden\\
neuer differencebased modus \\

