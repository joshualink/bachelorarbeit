\chapter{Zusammenfassung und Ausblick}\label{ch:zusammenfassung}
Im folgenden wird zusammengefasst zu welchen Schwierigkeiten oder Besonderheiten es bei den verschiedenen Phasen, sowie den verschiedenen Kapiteln der Arbeit gab. Danach werden die Ergebnisse der Arbeit zusammengefasst. Zuletzt wird ein Ausblick gegeben in welche Richtungen man noch weiter entwickeln und forschen kann.
\section{Besonderheiten}
Beim schreiben der Grundlagen war die eigentliche Schwierigkeit zu differenzieren, welche Informationen über das bestehende PSM-System wichtig für das Verständnis der Arbeit sind, welche nur angeschnitten werden müssen, sodass bei Interesse in die eigentlichen Arbeit über das PSM-System(\cite{gehrung14})  nachgelesen werden kann, und welche Informationen aus dieser Arbeit ganz rausgelassen werden können.\smallskip\\
Die Entwicklung eines Konzepts war sehr mit dem zweiten Teil der Implementierung verbunden, indem der Algorithmus implementiert wurde. Ein erstes Konzept lies sich relativ schnell entwickeln, doch sobald dieses implementiert war, wurden dessen Schwächen bei der praktischen Anwendung offenkundig und das Konzept musste geändert werden. Diesen Prozess machte das Konzept mehrmals durch, bis das in der Arbeit vorgestellte Konzept stand.\smallskip\\
Im ersten Teil der Implementierung steckt mit Abstand der meiste Arbeitsaufwand der Arbeit. Zu Beginn musste ich mich komplett in das bestehende System einarbeiten. Da das PSM-System kaum genutzt wurde, seit es 2014 entwickelt wurde, dauerte es etwas, das bestehende System überhaupt zum laufen zu bringen. Außerdem mussten Klassen quer über alle Module geädert werden um den gewollten Klassenaustausch durchzuführen. Statt einen Konverter zu verwenden, der die entsprechenden Klassen in der Mitte des Systems umformt und diesen Konverter mit jeder aufgebesserten Klasse im System weiter zu schieben, wurde das System im ganzen bearbeitet und erst nach Beendigung des Klassenumtauschs der Debug vorgenommen.\smallskip\\
Der Evaluationsprozess war hingegen eher intuitiv. Verschiedene Experimente wurden durchgeführt, dokumentiert und interpretiert. Schwierig fiel bei diesem Kapitel beziehungsweise der Auswertungsarbeit, sich seiner Interpretationsansätze sicher zu sein.\smallskip\\
Im ganzen gesehen habe ich im Zuge der Arbeit gelernt, wie ein Forschungsprojekt aussehen kann, welches in Einzelkomponenten aufgeteilt ist, sodass viele Mitarbeiter parallel daran arbeiten können, wie man wissenschaftlich schreibt, wie man ein Thema gut recherchiert und Paper zu diesem findet und wie man sich eigenständig motiviert und eigenständig arbeitet.\smallskip\\
\section{Zusammenfassung}
Im Zuge der Arbeit wurde ein neuer differenzbasierter Modus für das PSM-System entwickelt. Dieser arbeitet dichter an den Daten als das die schon vorhandenen Modi, indem er die erkannte Evidenz direkt mit den gelernten Daten abgleicht. Dies führt zu eindeutiger Szenenerkennung, aber auch zu weniger Robustheit gegenüber fehlenden Objekten. Allerdings kann man die Parametisierung dieses Modus noch weiter optimieren, um dieser geringen Robustheit entgegen zu wirken.\smallskip\\
Außerdem wurde das PSM-System restauriert und umgebaut. Vor der vorliegenden Arbeit wurde das System länger nicht genutzt und Ziel meiner Arbeit und der parallel laufenden Arbeit von Nikolai Gaßner(\cite{gassner17}) war es, das PSM-System so aufzurüsten, dass es wieder konkurrenzfähig zu dem ISM-System wird, an welchem schon weiter geforscht wurde. Das PSM-System sollte aktiv genutzt werden um Szenenwahrscheinlichkeiten zu erfragen, selbst wenn es nur passiert, um die Ergebnisse mit denen des ISM-Systems zu vergleichen. Bei diesem Vergleich sollten mehrere Modi des PSM-Systems genutzt werden, da es nun mehrere aussagekräftige Modi gibt.\smallskip\\
\section{Ausblick}
Die wahrscheinlichkeitsbasierte Szenenerkennung ist noch nicht perfekt und kann noch in vielen Hinsichten Optimiert werden. Man kann das bestehende System, um weitere Modi erweitern oder die bestehenden Modi in ihrer Laufzeit, in ihrer Speicherplatznutzung und ihrer Erkennung optimieren. Konkret am neu entwickelten Modus ließe sich die Constellation Model Nutzung einbauen. Das würde bedeuten, dass die Erkennung nicht mehr jede Relation zwischen allen Objekten nutzt, sondern nur einzelne signifikante Relationen überprüft. Bei einem gedeckten Frühstückstisch mit Cornflakes, Milch, Schüssel und Löffel fiele zum Beispiel auf, dass bei vielen unterschiedlichen Instanzen dieser Szene der Löffel fast immer in der gleichen Relation zur Schüssel bliebe. Diese Relation wäre dann signifikanter für die Szene als die anderen.\smallskip\\
Desweiteren kann man den differenzbasierten Modus weiterentwickeln, sodass der Modus robuster gegenüber fehlender Objekte wird und damit die Vergleichbarkeit zu anderen PSM-Modi größer ist. Wenn eine Szene nicht komplett aufgeabut ist, sollte die Erkennung sie nach Möglichkeit erkennen, sodass ein Serviceroboter mit der Erkennung die Szene selbstständig komplettieren kann.

