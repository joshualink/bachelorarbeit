\chapter{Motivation und Problemstellung}\label{ch:motivation}
Das Kapitel geht in Motivation auf die Relevanz des Themas und der vorliegenden Arbeit ein und in Fokus der Arbeit auf die Problemstellung die bearbeitet wird, sowie diverse Einschränkungen und Annahmen die für die Arbeit festgelegt sind.

\section{Motivation}
Wie schon in der Einführung erwähnt ist es elementar wichtig, dass man eine zuverlässige Szenenerkennung und ein gutes Kontextverständnis schafft um den Robotern die Möglichkeit zu bieten, sinnvoll mit ihrer Umwelt zu interagieren. Mit einer präzisen Wahrscheinlichkeitseinschätzung wie die momentane Umgebung beschaffen ist, lässt sich abschätzen welche Aufgaben es zu bewältigen und welche Probleme zu lösen gilt. Um dies zu gewährleisten muss man in das System möglichst viele Referenzdaten einspeisen, damit es jeden vorhanden Kontext erkennen kann. Szenen werden zu diesem Zweck aufgebaut un der Erkennung als neue Szene vorgestellt. Jede Szene die die Szenenerkennung auf diese Weise lernt, hilft das Chaos der sie umgebenen Objekte mehr und mehr zu interpretieren und einzuordnen.\smallskip\\
Die Szenenerkennung nutzt also die Daten die sie zur Verfügung gestellt bekommt, bereits gelernte Szenen wiederzuerkennen. In dem bereits vorhandenen PSM(Probabilistic Scene Model)-System werden die Daten pro Szene zu einem Modell zusammengefasst, bei dem Auffällige Zusammenhänge berücksichtigt werden und scheinbar nicht miteinander in Verbindung stehende Objekte voneinander gelöst betrachtet werden. Zum Beispiel findet man eine Computermaus signifikant häufig vor dem Computerbildschirm und nie dahinter, allerdings kann dabei das räumliche Verhältnis der Maus zur Tastatur stark variieren. Die Maus wäre in diesem Beispiel manchmal direkt neben der Tastatur, manchmal dichter beim Bildschirm eben so oft weiter entfernt. In diesem Fall würde möglicherweise die Relation zwischen Maus und Tastatur wegfallen und nur jeweils das räumliche Verhältnis zum Bildschirm betrachtet werden. Dadurch gibt es Vorteile in der Laufzeit und möglicherweise auch eine signifikantere Erkennung, allerdings findet natürlich auch ein Informationsverlust statt, der zu Fehlern führen kann. In dieser Arbeit wird ein Ansatz getestet, der dichter an den erhaltenen Daten arbeitet.\smallskip\\
In Abbildung \ref{img:relation} sieht man wie das Modell eine Relation zwischen zwei Objekten aufgrund der erhaltenen Daten erstellt. Die roten Pfeile beschreiben hier die jeweiligen Relationen, die aus den Daten berechnet wurden und die Ellipsen zeigen, welchen Bereich das Modell als Basis für die Wahrscheinlichkeitsabschätzung benutzt.

\begin{figure}
	\centering
	\includegraphics[width=10cm]{bilder/relation.pdf}
	\caption{Beispiel: Relative Position eines Objekts zu einem anderen \cite{gehrung14}}
	\label{img:relation}
\end{figure}

\section{Fokus der Arbeit}
Ziel der vorliegenden Arbeit ist es das PSM-System zu überarbeiten und einen neuen Modus der Szenenerkennung im PSM-System zu entwickeln. Dieser Modus soll alternativ zu den bereits vorhandenen Modi auswählbar sein und das System erweitern. Außerdem soll sowohl die Positionierung als auch die Rotation der erkannten Objekten mit den bekannten Szenen verglichen werden um eine Wahrscheinlichkeitsabschätzung auszugeben, welche die Wahrscheinlichkeit aller möglichen Szenen ausgibt sowie auch die Wahrscheinlichkeit, dass es sich um keine der Szenen  handelt.\smallskip\\
Sei M das Szenenmodell, welches Szene S darstellt, D die Daten auf denen M basiert und sei O eine Menge an Objekten, die momentan von einer Objekterkennung erkannt werden. Dann sei f(O, M,  D) eine Funktion, die die Objekte, die Daten und das Modell annimmt und daraus die Wahrscheinlichkeit von Szene S berechnet, dass sie in O enthalten ist. Es soll also eine neue Funktion f entstehen, welche dicht an den Daten arbeitet, die von der zu testenden Szene S vorhanden sind.\smallskip\\
\begin{figure}
	\centering
	\includegraphics[width=14cm]{bilder/SzenenBeispiel.pdf}
	\caption{Zwei Beispielszenen - Relative Lage der Objekte ändert die Szene \cite{gehrung14}}
	\label{img:relativeLage}
\end{figure}
Um eine interaktive Szenenerkennung zu schaffen ist es sinnvoll das System datengetrieben zu programmieren. Datengetriebene Entwicklung zeichnet sich dadurch aus, dass sich der Programmfluss ändert aufgrund der Daten die das System während der Laufzeit als Eingabe erhält. Das PSM-System wurde bereits datengetrieben programmiert und in der zu dieser Arbeit gehörigen Implementierung wurde auch am datengetriebenen Entwicklungsmodell festgehalten.\smallskip\\ 
Da die Erkennung innerhalb des PSM-Systems nutzbar sein soll sind Eingabe- und Ausgabeschnittstellen sowie die Visualisierung vorgeschrieben und sollen für gute Vergleichbarkeit denen des vorhandenen Systems entsprechen.\smallskip\\
Die Arbeit ist wie folgend strukturiert. In Kapitel \ref{ch:grundlagen} Grundlagen wird auf die Grundkenntnisse eingegangen die man braucht um den Rest der Arbeit gut zu verstehen. Es werden in diesem Kapitel die für die Arbeit wichtigen Komponenten des bestehenden PSM-System erklärt. In Kapitel \ref{ch:konzept} Konzept wird das theoretische Konzept thematisiert, welches zum Algorithmus geführt hat, dass den neuen Modus vom alten System unterscheidet. Außerdem wird der Algorithmus selbst erläutert. In Kapitel \ref{ch:implementierung} Implementierung wird die Software beschrieben und dokumentiert die im Zuge dieser Arbeit entstanden ist sowie die Änderungen die am bestehenden PSM System vorgenommen wurden. Kapitel \ref{ch:evaluation} Evaluation beschreibt die durchgeführten Experimente und interpretiert sie. In Kapitel \ref{ch:zusammenfassung} Zusammenfassung und Ausblick wird die Arbeit noch einmal zusammen gefasst und ein Ausblick darauf gegeben in welche Richtung man das bestehende System weiter entwickeln und auf welche Weise man  möglicherweise die Szenenerkennung noch weiter verbessern kann. Am Schluss stehen die Quellen welche zur Recherche für diese Arbeit genutzt wurden.

