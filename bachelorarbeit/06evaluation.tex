\chapter{Evaluation}\label{ch:evaluation}
\section{Experiment 1: Frühstück}
\begin{figure}
	\centering
	\includegraphics[width=8cm]{bilder/allobjects.pdf}
	\caption{Alle Objekte, die bei Experiment 1 genutzt werden \cite{gassner17}}
	\label{img:allobjects}
\end{figure}
\begin{figure}
	\centering
	\includegraphics[width=14cm]{bilder/fruehstueckszenen.pdf}
	\caption{Fr{\"u}hst{\"u}ckszenen Beispiel \cite{gassner17}}
	\label{img:fruehstueckexample}
\end{figure}
\begin{figure}
	\centering
	\includegraphics[width=14cm]{bilder/evaluationfruehstueckszenen.pdf}
	\caption{Fr{\"u}hst{\"u}ckszenen \cite{gassner17}}
	\label{img:fruehstueck}
\end{figure}
\section{Experiment 2: Büro}
\begin{figure}
	\centering
	\includegraphics[width=12cm]{bilder/allobjectsbuero.pdf}
	\caption{Alle Objekte, die bei Experiment 2 genutzt werden \cite{gassner17}}
	\label{img:allobjectsbuero}
\end{figure}
\begin{figure}
	\centering
	\includegraphics[width=12cm]{bilder/evaluationbueroszene.pdf}
	\caption{Angelernte Daten der B{\"u}roszene \cite{gassner17}}
	\label{img:buero}
\end{figure}
\begin{figure}
	\centering
	\includegraphics[width=10cm]{bilder/buerowahrscheinlichkeit.pdf}
	\caption{Erkennungswahrscheinlichkeiten der B{\"u}roszene}
	\label{img:buerowahrscheinlichkeit}
\end{figure}
Viele Bilder, beschreiben Daten\\
Text interpretiert\\
Fazit am Ende \\
