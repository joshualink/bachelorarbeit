\chapter{Einführung}\label{ch:einleitung}

\todo{Inhalt}\\
\begin{deprecated}
\cite{davis93}

A general comment about punning: Punning has been introduced to OWL 2 DL to
allow for some greater flexibility in naming entities. Now you can give the
same name to both an individual and a class or, if you like, to both a class
and a property, and the ontology will still count as a valid OWL 2 DL
ontology. This is, however, entirely a syntactic aspect. Semantically, the two
equally named entities are completely unrelated. Citing the old Next Steps for
OWL paper, which gave an overview about the original OWL 1.1 proposal, the
forerunner of OWL 2 DL:

OWL 1.1 uses a (weak) form of meta-modelling called punning. In punning, names
can be used for several purposes; for example, Person can at the same time be
the name of a class and the name of an individual. The different uses of a
name are, however, completely independent, and from a semantic point of view
they can be thought of as separate names, e.g., Person-the-Class and
Person-the-Individual.

So in your example, you cannot infer anything new from giving the same name to
both a class and a property that you would not already receive from using
distinct names for these entities. In particular, the two entities do /not/
become equal by giving the same name to them. This would, in fact, be
impossible, since in OWL DL a class represents a subset of the universe of
discourse, while a property represents a binary relation over that universe -
they are completely different kinds of entities.

In addition, note that punning is new only for OWL 2 /DL/ and all the OWL 2 DL
profiles. In OWL Full, just as in RDFS, it has always been possible to give
the same name to entities from different entity types without any
restriction. More, in OWL Full the same name always denotes the same thing,
with the corresponding semantic consequences. This is so since in OWL Full,
again as in RDFS and unlike OWL DL, classes and properties do not directly
stand for sets and binary relations, but instead they are individuals with
sets or binary relations associated to them. So, since classes and properties
are actually individuals, they may become equal, and in fact do so when being
equally named.

However, in the specific case of your class/property example, you will not
even get semantic consequences of the kind you ask for from OWL Full. The
reason is that, while an individual can really have both a set and a binary
relation associated to it (i.e., the individual can be both a class and a
property), there will be no relationship between that set and that relation --
quite similar as in the case of OWL 2 DL punning. For example, the individual
can on the one hand be the empty class, while on the other hand it is a
non-empty binary relation, e.g. by having a triple with the individual as its
predicate in the ontology.
\end{deprecated}
