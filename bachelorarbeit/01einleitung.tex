\chapter{Einführung}\label{ch:einleitung}

In der Robotik ist die Servicerobotik wohl der Forschungsbereich, welcher den größten Alltagsbezug für den Menschen hat, da er sich mit der Entwicklung und 
Weiterentwicklung von autonomen Robotern beschäftigt, welche dem Menschen im Alltag assistieren.  Man findet mittlerweile Roboter im Privaten, die das Putzen, Staubsaugen oder Rasenmähen übernehmen, in der Industrie, bei Montage und Fertigung, sowie auch in der Medizin, als Pflegehilfe, Botengänger oder Assistent.\\
Allerdings müssen die Roboter ihre Umwelt für komplexere Aufgaben so präzise wie möglich wahrnehmen und verstehen. Sie könne Aufgaben übernehmen bei denen sie gezielt Objekte umfahren, suchen und auch aufnehmen und benutzen. Dieser Funktionsumfang kann mit dem Prizip Programmieren durch Vormachen (PdV) ermöglicht werden, bei dem die Roboter Objekte und Tätigkeiten ihrer Umgebung kennen lernen, wieder erkennen und nachahmen können. So lässt sich die hohe Komplexität umgehen, die die manuelle Programmierung vieler Aufgaben mit sich bringen würde.\\
Um tatsächlich selbstständige Serviceroboter zu schaffen muss man aber noch zu einer Objekterkennung ein Kontextverständnis hinzufügen. Die Roboter müssen erkannte Objekte in einen Zusammenhang bringen, um die dadurch resultierenden Aufgaben zu verstehen. Zum Beispiel hat ein Teelöffel, welcher neben einer Tasse Tee liegt eine andere Aufgabe zu verrichten, als wenn er neben einem Becher Joghurt platziert ist. Nur am Kontext lässt sich dort entscheiden warum im einen Fall umgerührt und im anderen gelöffelt wird. Ebenso wäre ein Stück Butter verschieden zu verwenden, wenn es auf einem Frühstückstisch steht als wenn es mit anderen Zutaten neben einer Rührschüssel vorkommt.\\
Somit braucht man eine Szenenerkennung, welche zuverlässig die Objekte erkennen und ihren jeweiligen Kontext verstehen und einschätzen kann. Diese Erkennung ist nicht immer eindeutig, da der eben genannte Löffel ebenso zwischen einem Becher Joghurt und einer Tasse Tee liegen könnte, deshalb bietet es sich an mit Wahrscheinlichkeitsabschätzungen des vorliegenden Kontexts zu arbeiten.


